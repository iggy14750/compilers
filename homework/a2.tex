\documentclass{article}
\author{Isaac B Goss}
\title{Assignment 2}

\usepackage{amsmath}
\usepackage{amsthm}
\usepackage{enumitem}
\usepackage[margin=0.8in]{geometry}
\usepackage{graphicx}

% ============ USED FOR OUR FORMAT ============
\newtheorem{thm}{Claim}
\providecommand{\prob}[1]{\section*{Problem #1}}
\providecommand{\soln}{\textbf{Solution: }}
\providecommand{\image}[1]{
    \begin{center}
        \includegraphics%[width=0.95\textwidth]
            {#1}
    \end{center}
}
\providecommand{\tightlist}{
    \setlength{\itemsep}{0pt}\setlength{\parskip}{0pt}
}

% ============ USED FOR CODE LISTINGS ============
\usepackage{listings}
\usepackage[usenames,dvipsnames,svgnames]{xcolor}
\definecolor{javagreen}{rgb}{0.25,0.5,0.35}
\lstset{
    basicstyle   = \footnotesize,
    commentstyle = \color{javagreen},
    frame        = single,
    language     = C,
    stringstyle  = \color{orange},
    numbers      = left,
    showstringspaces=false,
    deletekeywords = {len, max, format, min},
    morekeywords = {yield, function, then, do, to},
    keywordstyle = \color{blue},
    escapeinside = {(*} {*)}
}

% ========== TEMPLATE FOR GRAMMARS ===========
% \begin{center}
%     \texttt{
%         \begin{tabular}{ll}
%             A $\to$ & aB | aC \\
%             B $\to$ & aC | bE \\
%             C $\to$ & bD      \\
%             D $\to$ & aC | bE \\
%             E $\to$ & bF      \\
%             F $\to$ & bG      \\
%             G $\to$ & bG | $\epsilon$
%         \end{tabular}
%     }
% \end{center}


\begin{document}
\maketitle

    \prob{1}
    Let's begin with encoding precedence.
    \begin{center}
        \texttt{
            \begin{tabular}{ll}
                E $\to$ & E `$||$' F \textbar{} F   \\
                F $\to$ & F `\&\&' G \textbar{} G   \\
                G $\to$ & `!' H \textbar{} H        \\
                H $\to$ & `true' \textbar{} `false' \\
            \end{tabular}
        }
    \end{center}

    And then left factoring that.
    \begin{center}
        \texttt{
            \begin{tabular}{ll}
                E  & $\to$ F E' \\
                E' & $\to$ `$||$' F E' \textbar{} $\epsilon$ \\
                F  & $\to$ G F' \\
                F' & $\to$ `\&\&' G F' \textbar{} $\epsilon$ \\
                G  & $\to$ `!' H \textbar{} H \\
                H  & $\to$ `true' \textbar{} `false' 
            \end{tabular}
        }
    \end{center}

    \prob{2}
    First, I found the First and Follow sets of the non-terminals.

    \begin{center}
        \begin{tabular}{l | l | l}
            \textbf{Non-Terminal} & \textbf{First Set} & \textbf{Follow Set}\\
            \hline
            E  & !, true, false   & \$             \\
            E' & $\epsilon$, $||$ & \$             \\
            F  & !, true, false   & $||$, \$       \\
            F' & $\epsilon$, \&\& & $||$, \$       \\
            G  & !, true, false   & \&\&, $||$, \$ \\
            H  & true, false      & \&\&, $||$, \$ \\
        \end{tabular}
    \end{center}

    Then used the given algorithm to build the parse table.
    \begin{center}
        \begin{tabular}{|l||l|l|l|l|l|l|}
            \hline
            & true & false & \&\& & $||$ & ! & \$ \\
            \hline \hline
            E  & E $\to$ F E' & E $\to$ F E'  &                    &                    & E $\to$ F E' &                   \\ \hline
            E' &              &               &                    & E' $\to$ $||$ F E' &              & E' $\to \epsilon$ \\ \hline
            F  & F $\to$ G F' & F $\to$ G F'  &                    &                    & F $\to$ G F' &                   \\ \hline
            F' &              &               & F' $\to$ \&\& G F' & F' $\to \epsilon$  &              & F' $\to \epsilon$ \\ \hline
            G  & G $\to$ H    & G $\to$ H     &                    &                    & G $\to$ ! H  &                   \\ \hline
            H  & H $\to$ true & H $\to$ false &                    &                    &              &                   \\ \hline
        \end{tabular}
    \end{center}

    \pagebreak
    \prob{3}
    The following chart contains information on the parse of the given input string.
    \begin{center}
        \begin{tabular}{l|l|l}
            \textbf{Stack} & \textbf{Input} & \textbf{Production/Action} \\ \hline
            \$ E            & ! false $||$ true \&\& false & E $\to$ F E'      \\
            \$ E' F'        & ! false $||$ true \&\& false & F $\to$ G F'      \\
            \$ E' F' G      & ! false $||$ true \&\& false & G $\to$ ! H       \\
            \$ E' F' H !    & ! false $||$ true \&\& false & pop !             \\
            \$ E' F' H      & false $||$ true \&\& false   & H $\to$ false     \\
            \$ E' F' false  & false $||$ true \&\& false   & pop false         \\
            \$ E' F'        & $||$ true \&\& false         & F' $\to \epsilon$ \\
            \$ E'           & $||$ true \&\& false         & E' $\to$ $||$ F E'\\
            \$ E' F $||$    & $||$ true \&\& false         & pop $||$          \\
            \$ E' F         & true \&\& false              & F $\to$ G F'      \\
            \$ E' F' G      & true \&\& false              & G $\to$ H         \\
            \$ E' F' H      & true \&\& false              & H $\to$ true      \\
            \$ E' F' true   & true \&\& false              & pop true          \\
            \$ E' F'        & \&\& false                   & F' $\to$ \&\& G F'\\
            \$ E' F' G \&\& & \&\& false                   & pop \&\&          \\
            \$ E' F' G      & false                        & G $\to$ H         \\
            \$ E' F' H      & false                        & H $\to$ false     \\
            \$ E' F' false  & false                        & pop false         \\
            \$ E' F'        & \$                           & F' $\to \epsilon$ \\
            \$ E'           & \$                           & E' $\to \epsilon$ \\
            \$              & \$                           & Accept            \\
        \end{tabular}
    \end{center}
\end{document}

























